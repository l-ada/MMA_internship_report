\documentclass[a4paper,10pt]{report}
\usepackage[utf8]{inputenc}

% Title Page
\title{Title}
\author{Łukasz Adamowicz}

\usepackage{mathcommands}


\begin{document}
\maketitle

\begin{abstract}
\end{abstract}



\section{ Lab presentation}
SciAi groups research focus is on ...

 \section{Math part}
Notation: Let $E(\theta, M ,p)$ denote the energy model, with $\theta$ being model parameters, $M$ the molecule and $p\in \R^n$ density coefficients.
TK $p$ DIMENSION DEPENDS ON N $M$

 \subsection{Implicit}
1. Implicit funciton theorem

The goal of the project was to investigate and minimize the following loss
\begin{equation}
 \min_\theta \sum_i \mathcal{L}(p_{\theta}^{M_i}),
\end{equation}
where
\begin{equation}
  p_\theta^{M_i}:=\argmin_{p : \inner{w}{p}=N_i} E(\theta,M_i,p)
\end{equation}
and $\mathcal{L}_{M_i}(p) = \frac{1}{2}\mse{p-p_{M_i}}$ is the standard $L_2$ loss.

In the rest of the report I will drop index $M_i$, since it the loss is a sum of individual losses for each molecule and it will notably simplify the notation.
Therefore I will write $E(\theta,p)$ instead of $E(\theta,M,p)$ and the same for $p_\theta$, $\mathcal{L}(p)$ and so on.

This falls under the domain of \textbf{bilevel optimization}.

The main problem lies in computing the gradient of $\mathcal{L}(p_\theta)$


\subsubsection{ IFT for one molecule}
Let $w$ be a non-zero vector.
We consider the constrained problem of minimizing

\begin{equation}
 \min_\theta \mathcal{L}(p_\theta),
\end{equation}
where
\begin{equation}
 p_\theta = \argmin_{p: \inner{p}{w}=N} E(\theta, p)
\end{equation}

This can be accomplished by using implicit function theorem

TK ADD STATEMENT

TK DISCLAIMERS ABOUT APPLICABILITY, IT BEING A FUNCTION ETC, JACOBIAN

TK DOES IT EVEN WORK IF IT'S NOT A FUNCTION
\subsubsection{Derivation of the gradient}
TK ADD SECTION FROM THE BLOG
Before, we had
$$\nabla_p E(w,p_w) = 0.$$
This time it changes to
$$P\big( \nabla_p E(w,p_w)\big)=0,$$
where $P = Id - \frac{vv^T}{v^Tv},$ ie. operator $P$ is just a projection onto subspace $V_0 = \{p: \langle p,v \rangle = 0\}.$


For any $u\in V_0$ we have $\langle \nabla_p E(w,p_w), u \rangle = 0$, since $p_w$ is the minimum of $E(w,p)$ restricted to $V$.


But this means exactly that $P\big( \nabla_p E(w,p_w)\big)=0,$ since we can try $u = P\big( \nabla_p E(w,p_w)\big)$.

Moreover $\langle \nabla_p E(w,p_w), P\big( \nabla_p E(w,p_w)\big) \rangle = \langle P\big( \nabla_p E(w,p_w)\big) , P\big( \nabla_p E(w,p_w)\big) \rangle$.

Let's denote $F(w,p) =P (\nabla_pE(w,p)).$ Then the previous condition can be written as
$$F(w,p_w) = 0.$$
$$\frac{d p_w}{dw} = - \bigg(\frac{\partial F}{\partial p}\bigg)^{-1}  \frac{\partial F}{\partial w}\bigg|_{p=p_w, w=w}.$$
But the jacobian $\frac{\partial F}{\partial p}$ is not inversible. It can't be, since $F$ projects gradient onto subspace of lower dimension.

Therefore, we also have to reformulate solving for $y$ vector

$$y =- 2*(p_w-p_{true}) \cdot J^{-1}$$
as a minimal square problem
$$y = \underset{u}{\mathrm{argmin }}\|uJ + 2(p_w-p_{true})\|^2$$ and adjust accordingly, while computing it.


$$ \frac{d}{dw} \nabla_p E(w,p_w) = 0.$$
 Let's denote $F(w,p) = \nabla_p E(w,p)$ and let $J = J_{F}(p_w)=\frac{\partial F}{\partial p}$ be the jacobian of $F$ at $p = p_w$.
We obtain

$$\frac{d}{dw}F(w,p_w) = \frac{\partial F}{\partial w} + \frac{\partial F}{\partial p}\frac{d p_w}{dw}= 0 .$$

**Remark** : We treat $p_w$ as a function of $w$, which may not be applicable for $E$.
For example, there might be multiple minima.

In the end, we obtain

$$ \frac{d p_w}{dw} = - J^{-1} \frac{\partial F}{\partial w}.$$

Thus,

$$\frac{d J}{dw} = 2*(p_w-p_{true})\cdot \frac{d p_w}{dw} = 2*(p_w-p_{true})\cdot (-J^{-1}) \frac{\partial F}{\partial w}.$$

The usual way to deal with this equation is to solve

$$y =- 2*(p_w-p_{true}) \cdot J^{-1}$$
or equivalently
$$ yJ = -2*(p_w - p_{true}),$$ which is just a linear system of equations.

3. Lagrangian approach?

TK COME BACK WHEN YOU ACTUALLY DO IT BY HAND
 \begin{equation}
\min_{\lambda,\theta} L
 \end{equation}

4. Proof of IFT with constraints converging

5. Error analysis for both approximate fixed point and solving lineare equation
\subsection{ Equilibrium propagation}

\subsubsection{Base version}

Let's denote
\begin{equation}
H(\theta, p, \beta) := \beta \mathcal{L}(p) + E(\theta, p),
\end{equation}
where $\beta \in \R$.
 This is called \textbf{total energy} in \cite{eqprop}. Let's also denote
\begin{equation}
p_{\theta}^{\beta}:=\argmin_p H(\theta, p, \beta)
\end{equation}
and
\begin{equation}
 G(\theta, \beta) = H(\theta, p_\theta^\beta, \beta).
\end{equation}

Since $H(\theta,p,0) = E(\theta,p)$ we have $p_{\theta}^{0}=p_{\theta}$.

Statement:

\begin{equation}
 \frac{d}{d\theta} \mathcal{L}(p_\theta) = \lim_{\beta \to 0} \frac{\pd{H}{\theta}(\theta, p_\theta^\beta, \beta)-\pd{H}{\theta}(\theta, p_\theta^0, 0) }{\beta}
\end{equation}



\begin{equation}
 \frac{d}{d\theta}\frac{d}{d\beta}G(\theta,\beta)\big|_{\beta=0} = \frac{d}{d\theta} \mathcal{L}(p_\theta)
\end{equation}






2. Proof

3. EqProp with constraints



4. Proof of convergence

5. Error margins
\section{ Implementation}
0. Technical details
    architecture
    training
    etc
1. Stability techniques

2. algorithm
\section{Results}


\nocite{*}
\bibliographystyle{plain}
\bibliography{references.bib}



\section{Appendices}
\subsection{Proof of equilibrium propagation formula}
\end{document}
